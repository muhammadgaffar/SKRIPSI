%-----------------------------------------------------------------------------%
\chapter{\babSatu}
%-----------------------------------------------------------------------------%

%-----------------------------------------------------------------------------%
\section{Latar Belakang}
%-----------------------------------------------------------------------------%
Model Drude \cite{drude} salah satu model pertama yang berhasil mendeskripsikan sifat-sifat fisis dari sebagian besar logam. Misalnya saja, model Drude dapat menjelaskan resistivitas residu pada temperatur rendah, suseptibilitas magnet yang tidak bergantung dengan temperatur, dan kapasitas panas yang bergantung  secara linear terhadap temperatur. Model Drude dijelaskan dengan mempertimbangkan elektron sebagai gas elektron yang tidak berinteraksi. Selanjutnya, model elektron hampir bebas (\textit{nearly free electron}) atau teori struktur pita energi\cite{ashcroft-mermin} memperbaiki model Drude, dimana mempertimbangkan potensial periodik dari kisi atomik. Hal ini tidak hanya dapat menjelaskan logam dengan baik, tetapi juga menjadi pendekatan yang kuat untuk menjelaskan material non-logam, yakni isolator dan semikonduktor. Perhitungan struktur pita energi telah terus diperluas pada situasi-situasi yang lebih realitis untuk elektron banyak-benda seperti memperhitungkan \textit{exchange mechanism} dan interaksi coulomb yang didekati secara medan rata-rata. Pendekatan medan rata-rata juga diketahui sebagai teori Hartree-Fock dimana fungsi gelombang benda-banyak ditulis dalam determinan slater. Namun, menghitung dengan determinan slater bisa menjadi sangat mahal dimana membutuhkan jangka waktu yang lama dikarenakan jumlah elektron yang diperhitungkan sangat banyak ($\sim 10^{22}$) pada situasi material asli. Metode alternatif kemudian dikembangkan oleh W.Kohn dan L.J.Sham\cite{kohn-sham}, yang dimana karena pendekatannya dapat ditulis energi dasar dari elektron banyak-benda sebagai fungsional densitas elektron. Pendekatan ini diketahui sebagai \textit{density functional theory} (DFT), yang dimana telah sukses memprediksi energi dasar berbagai material dalam beberapa dekade terakhir.

Walaupun DFT terbilang sangat sukses dalam menjelaskan sifat fisis dari berbagai material, masih terdapat kelompok material yang belum dapat terjelaskan dengan pendekatan tersebut, dan membutuhkan teori baru. Di tahun 1937, de Boer dan Verwey\cite{boer} memperlihat dalam eksperimennya bahwa adanya sifat isolator pada logam transisi oksida seperti NiO dan CoO walaupun orbital-$d$ yang belum terisi penuh di material tersebut diprediksi akan memiliki sifat logam dalam sudut pandang teori pita energi. Nevil Mott\cite{mott} berpendapat bahwa terdapat interaksi tolak-menolak Coulomb antar elektron yang menghalangi terjadinya sifat logam, sehingga material tersebut sebagai mott isolator. Mott isolator dapat dilihat pada logam \textit{Fermi liquid} saat adanya perubahan besar interaksi sehingga terjadi transisi logam-isolator (\textit{metal-insulator}) yang juga diketahui sebagai Transisi Mott. Logam yang memiliki Transisi Mott menunjukkan adanya resistivitas dan koefisien kapasitas panas yang tinggi. Tipe logam tersebut dikategorikan sebagai logam buruk (\textit{bad metal}) walaupun mereka memiliki sifat \textit{Fermi liquid}. Kategori material yang memiliki kontribusi interaksi Coulomb antar elektron yang tidak dapat diabaikan disebut sebagai material terkolerasi kuat.

Model Hubbard menjadi model paling sederhana yang dapat menangkap sifat fisis yang esensial pada material terkolerasi kuat dimulai dari transisi Mott-Hubbard logam-isolator\cite{mott-hubbard}, antiferomagnetik\cite{antiferomagnetik1}, dan superkonduktivitas $d$-\textit{wave}\cite{d-wave}. Meskipun minimnya solusi eksak dari model Hubbard, namun banyak terdapat pendeketan teoritis yang dilakukan secara aproksimasi. Salah satu metode yang paling ampuh dalam mengerjakan model Hubbard adalah \textit{Dynamical Mean-Field Theory}(DMFT)\cite{DMFT}. DMFT memiliki kesamaan dengan pendekatan medan rata-rata, keduanya memetakan masalah kisi banyak-benda ke site tunggal yang bersifat lokal. Metode ini menjadi eksak pada batas dimensi yang tak berhingga. 

DMFT pada dasarnya adalah memetakan kisi banyak-benda kuantum, seperti model Hubbard, ke model impuritas, dengan kondisi self-consistency. Hal ini menjadi sangat praktis dikarenakan pada batas dimensi tak berhingga \textit{self-energy} menjadi tidak bergantung dengan ruang momentum, dan hanya bergantung pada energi elektron, sehingga kuantitasnya bersifat lokal. Berbeda dengan DFT yang menggunakan densitas elektron sebagai salah satu kuantitas utamanya, DMFT dihitung pada formalisme fungsi Green. Model impuritas ini secara umum dapat diselesaikan banyak metode, diantaranya quantum Monte Carlo, Diagonalisasi Eksak, dan teori iterasi pertubasi. Namun, metode quantum Monte Carlo menggunakan metode komputasi yang sangat berat dikarenakan teknik samplingnya yang harus pada jumlah yang besar, sama halnya dengan Diagonalisasi Eksak yang membutuhkan sumber daya komputasi yang besar dikarenakan perhitungan total basis yang sangat besar pada situasi material asli sehingga ini biasanya dikerjakan pada sistem dengan ukuran terbatas, sedangkan teori iterasi pertubasi fluktuasi kuantum didekati secara pertubatif yang dimana ini menggunakan sumberdaya komputasi yang ringan namun tidak dapat menjelaskan pada daerah dengan interaksi coulomb yang kuat dimana pendekatan pertubasi tidak berlaku. Hingga saat ini masih dilakukan pengembangan perhitungan model impuritas yang akurat namun membutuhkan sumberdaya komputasi yang tidak besar.


%-----------------------------------------------------------------------------%
\section{Batas Permasalahan}
%-----------------------------------------------------------------------------%
Dibutuhkan pengembangan metode penyelesaian impuritas pada DMFT yang cukup akurat dan efektif namun menggunakan sumberdaya komputasi yang ringan. Metode ini dikembangkan atas dasar pengembangan lebih jauh dari medan rata-rata yang diaplikasikan dalam DMFT. Model impuritas medan rata-rata dan teori iterasi pertubasi akan digunakan sebagai metode pembanding untuk melihat sejauh mana metode baru ini bekerja dengan baik dalam menampilkan sifat-sifat fisis dari model Hubbard. Disini akan dibatasi sifat-sifat fisis yang ditinjau, diantaranya yang akan dihitung adalah diagram fasa dari transisi logam-isolator, magnetisme, diagram fasa magnetisme, dan total jumlah keadaan energi elektron.

Dalam mengembangkan metode komputasi numerik yang baru, cukup dihitung pada model material yang sangat sederhana, tanpa harus menggunakan kasus material asli. Berikut akan digunakan model-model material non-fisis seperti kisi Bethe dan kisi 3D paling sederhana, yakni kubik. Hasil utama akan dievaluasi dari hasil literatur dan metode pembanding yang digunakan. Model material ini tetap akan masih menunjukkan sifat yang sama pada material asli, dikarenakan pada dasarnya hal utama yang menjelaskan sifat fisis material adalah hamiltonian itu sendiri, yakni model Hubbard.

Diharapkan metode ini dapat secara lebih akurat menampilkan sifat-sifat fisis material transisi logam-isolator terutama pada kasus dengan interaksi Coulomb yang kuat dimana metode teori iterasi pertubasi tidak dapat menangkap fisisnya dengan baik.


%-----------------------------------------------------------------------------%
\section{Tujuan Penelitian}
%-----------------------------------------------------------------------------%
\begin{itemize}
\item Mengembangkan metode model impuritas dan algoritma \textit{self-consistent} DMFT yang baru dikembangkan lebih jauh dari medan rata-rata yang memiliki beban komputasi yang relatif ringan.
\item Meninjau berbagai sifat-sifat fisis model Hubbard  dengan menggunakan model impuritas dan algoritma yang telah dikembangkan dari berbagai parameter-parameter fisis.
\end{itemize}

%-----------------------------------------------------------------------------%
\section{Metode Penelitian}
%-----------------------------------------------------------------------------%
Penelitian ini dimulai dengan mengembangkan formulasi analitik matematis model impuritas baru yang dikembangkan dari medan rata-rata. Pengembangan ini dilakukan dengan menambahkan suku fluktuasi yang bersifat semi-klasik pada okupansi setiap basis kisi. Secara kuantum fluktuasi ini akan nantinya akan direpresentasikan secara probabilitas yang akan digunakan untuk menghitung \textit{self-energy} pada sistem. Hasil \textit{self-energy} ini akan menjadi koreksi dari yang dihasilkan oleh teori iterasi pertubasi. 

Setelah dikembangkannya formulasi matematis dari \textit{self-consistency} model impuritas tersebut, akan diimplementasikan secara numerik untuk dilakukan perhitungan untuk melihat sifat-sifat fisis yang dihasilkan seperti jumlah keadaan energi elektron dan magnetisasinya, dengan melihatnya dari berbagai parameter fisis seperti energi kinetik elektron yakni \textit{hopping} dan interaksi Coulomb antar elektron.

Implementasi numerik akan ditulis dalam pemograman bahasa Julia dan menggunakan berbagai \textit{library} nya yang disediakan untuk memudahkan perhitungan dan proses penulisan numerik, khususnya menggunakan \textit{package} Aljabar Linear, Integrasi dan Differensiasi, serta Statistik. Untuk mempercepat proses perhitungan, implementasi numerik akan ditulis dalam bentuk paralel dalam bahasa Julia.

%-----------------------------------------------------------------------------%
\section{Sistematika Penulisan}
%-----------------------------------------------------------------------------%
Sistematika penulisan laporan adalah sebagai berikut:
\begin{itemize}
	\item Bab 1 \babSatu \\
	Bab 1 pada skripsi ini dimulai dengan menjelaskan latar belakang penelitian secara singkat dengan meninjau permasalahan yang dialami saat ini, selanjutnya dimulai dengan membatasi permasalahan yang akan dikerjakan dalam perhitungan dalam skripsi untuk menjawab tujuan penelitian secara efektif dan cukup, lalu dilanjutkan dijelaskan secara singkat bagaimana permasalahan tersebut diselesaikan dalam subab metodologi, dan di akhir dibahas bagaimana sistematika skripsi ini ditulis.
	\item Bab 2 \babDua \\
	Bab 2 akan dijelaskan secara lebih dalam dan detail model Hubbard dan teori-teori yang penting dalam material terkolerasi kuat. Selanjutnya akan tinjau perkembangan dari metode DMFT, dimulai dari konsep, penurunan matematis, dan model-model impuritas yang ada, terutama akan dibahas model impuritas medan rata-rata dan teori iterasi pertubasi. Terakhir akan dijelaskan mengapa model material yang sederhana seperti kisi Bethe dan kisi kubik 3D cukup untuk memperlihatkan fenomena material terkolerasi kuat.
	\item Bab 3 \babTiga \\
	Pengembangan algoritma \textit{self-consistent} yang dikembangkan dari medan rata-rata dengan mempertimbangkan fluktuasi okupansi akan dijelaskan pada Bab 3. Perhitungan-perhitungan besaran fisis yang akan dihitung akan juga dijelaskan pada bab ini.
	\item Bab 4 \babEmpat \\
	Analisis hasil dari algoritma \textit{self-consistent} yang baru akan dijelaskan pada bab 4, analisis dilakukan dengan metode perbandingan dari literatur dan perhitungan model impuritas yang sudah ada, yakni medan rata-rata dan teori iterasi pertubasi.
	\item Bab 5 \babLima \\
	Diakhir, bab 5 akan meyimpulkan kembali dari teori dan hasil yang telah didapat dari bab-bab sebelumnya, dan memberikan saran dan evaluasi yang lebih lanjut penelitian lebih lanjut.
\end{itemize}

