%-----------------------------------------------------------------------------%
\chapter*{\kataPengantar}
%-----------------------------------------------------------------------------%
Pertama-tama, saya ucapkan rasa syukur kepada Allah swt, dikarenakan kasih sayang dan
rahmat-Nya, saya dapat menyelesaikan skripsi tepat waktu. Proses skripsi yang ditujukan
untuk mendapatkan gelar sarjana sains ini tidak lepas berkat kontribusi dan dukungan 
banyak pihak. Secara khusus, saya ucapkan terima kasih kepada:

\begin{enumerate}
	\item Muhammad Aziz Majidi, Ph.D yang telah menjadi pembimbing saya dalam menjalani
	studi dan riset di peminatan fisika zat mampat. Berkat bimbingannya, saya mendapatkan
	banyak ilmu dan motivasi dalam menjalani skripsi saya.
	\item Choirun Nisaa Rangkuti, sebagai mentor riset saya yang telah membantu membimbing saya
	dalam penulisan komputasi numerik pada awal-awal riset skripsi ini dimulai. 
	\item Kepada kedua orangtua saya, Eddy Asnawi dan Zetriuslita, dan kedua adik saya, Muhammad Luthfi 
	Taufiqqurahman dan Muhammad Aziz Habiburrahim yang selalu mendoakan saya.
	\item Teman-teman anggota lab \textit{Theoretical Condensed Matter Physics} (TCMP) yang telah
	menemani dalam suka dan duka selama rentang pengerjaan riset skripsi. Terkhususnya angkatan 2015, 
	1) Wilwin sebagai teman yang sangat memotivasi saya mengejar hal-hal akademik dan diskusi mengenai fisika,
	2) Dedi Prakasa sebagai teman yang unik dan lucu, 3) Andes Rogata Parmonangan Tambunan sebagai teman seperjuangan dan senasib
	dalam studi fisika, dan 4) Nur Atikah Tadjuddin sebagai teman pengerjaan satu topik riset.
	\item Teman-teman fisika non-TCMP lainnya, 1) Benedictus Bayu Respati sebagai teman dan sahabat yang luar biasa
	dalam berdiskusi mengenai perkuliahan dan kegiatan non-akademis.	2) Muhammad Mahdi Ramadhan sebagai teman pertama dan sahabat yang kaya pikiran yang selalu penuh dengan kejutan disaat berdiskusi, 3) Jason Kristiano, Agya Sewara Alam, sebagai teman yang menemani saya dalam 	diskusi mengenai fisika, 4) Faidzal Adilla, Ahmad Fauzan Kamaluddin, Faiz Rambey, Naufal Praditya, Reyan Qowi, Eufrat Tsaqib, Rasyid Sulaeman, dan Nabilah Hana Luthfiyah sebagai teman-teman yang pernah menemani saya dalam berbagai kegiatan di kampus,	dan 5) Abyan Jadiddan, Feby Nirwana, dan Indrianita Lionardi sebagai teman yang menjadi inspirasi saya untuk terus berbuat baik, bersikap rendah hati. Terakhir tentu, semua angkatan Fisika 2015 yang menjadi angkatan yang dinamis dalam 	menemani saya dalam 4 tahun studi fisika.
\end{enumerate}
Saya juga berterima kasih kepada semua orang yang telah banyak mendewasakan diri saya dalam menjalani perkuliahan, baik
dalam akademis maupun non-akademis. Saya berharap hasil skripsi saya ini akan berguna bagi siapa saja yang membacanya.

\vspace*{0.1cm}
\begin{flushright}
Depok, 28 Juni 2019\\[0.1cm]
\vspace*{1cm}
\penulis

\end{flushright}