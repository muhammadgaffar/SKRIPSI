%
% Halaman Abstrak
%
% @author  Andreas Febrian
% @version 1.00
%

\chapter*{Abstrak}

\vspace*{0.2cm}

\noindent \begin{tabular}{l l p{10cm}}
	Nama&: & \penulis \\
	Program Studi&: & \program \\
	Judul&: & \judul \\
\end{tabular} \\ 

\vspace*{0.5cm}

\noindent Sistem terkolerasi kuat adalah sistem dimana skala energi untuk interaksi antar partikel tidak lagi dapat diabaikan. \textit{Dynamical Mean Field Theory}(DMFT) menjadi salah satu metode yang banyak dikenal sebagai metode ampuh untuk menjelaskan fisika dari sistem terkolerasi kuat. Disini kami mempelajari metode penyelesaian impuritas yang tersedia di DMFT untuk model Hubbard, model paling sederhana dalam sistem terkolerasi kuat. Dengan melihat berbagai keterbatasan metode penyelesaian impuritas dengan sumberdaya numerik yang murah, kami mengembangkan metode penyelesaian impuritas dengan numerik yang murah lainnya yang dikembangkan dari metode medan rata-rata dengan melibatkan fluktuasi okupasi sebagai kuantitas numerik. Dengan melihat efek fluktuasi, kami membandingkan hasil tersebut dengan metode penyelesaian impuritas lainnya, yakni medan rata-rata dan iterasi pertubasi teori yang dipelajari pada keadaan paramagnetik dan antiferomagnetik. Kami simpulkan bahwa fluktuasi okupasi belum sepenuhnya mampu menjadi metode penyelesaian impuritas yang baik, namun memberikan hasil yang menarik untuk digunakan sebagai koreksi dari iterasi pertubasi teori.\\

\vspace*{0.2cm}

\noindent Kata Kunci: \\ 
\noindent Sistem Terkolerasi Kuat, DMFT, Model Hubbard, Medan Rata-Rata, Iterasi Pertubasi Teori, Okupasi, Fluktuasi \\

\newpage