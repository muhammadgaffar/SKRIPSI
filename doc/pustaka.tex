%
% Daftar Pustaka 
% 

% 
% Tambahkan pustaka yang digunakan setelah perintah berikut. 
% 
\begin{thebibliography}{4}

\bibitem{drude}
{P. Drude. "Zur Elektronentheorie der Metalle."
\f{Ann. der Phys.} \bo{306 (3)}, 566 (1900)}

\bibitem{ashcroft-mermin}
{N. W. Ashcroft and N. D. Mermin. "Solid  State  Physics." Harcourt,Orlando (1976)}

\bibitem{kohn-sham}
{W. Kohn and L. J. Sham. "Self-Consistent Equations Including Exchange and Correlation Effects." \f{Phys. Rev.} \bo{A 140}, 1133 (1965)}

\bibitem{boer}
{J. H. de Boer and E. J. W. Verwey. "Semi-conductors with partially and with completely filled 3d-lattice bands." \f{Proc. Phys. Soc} \bo{49}, 59 (1937)}

\bibitem{mott}
{N. F. Mott and R. Peierls. "Discussion of the paper by de Boer and Verwey." \f{Proc. Phys. Soc} \bo{49}, 72 (1937)}

\bibitem{mott-hubbard}
{M. J. Rozenberg, G. Kotliar, and X. Y. Zhang. "Mott-Hubbard transition in infinitedimensions. II." \f{Phys. Rev. B} \bo{49}:10181–10193 (1994)}

\bibitem{antiferomagnetik1}
{J. E. Hirsch and S. Tang. "Antiferromagnetism in the Two-Dimensional Hubbard Model." \f{Phys. Rev. Lett} \bo{62}:591–594 (1989)}

\bibitem{d-wave}
{T. Yanagisawa. "Physics of the Hubbard model and high temperature supercon-ductivity." \f{JPCS} \bo{108}:012010 (2008)}

\bibitem{DMFT}
{A. Georges, G. Kotliar, W. Krauth, and M. J. Rozenberg. "Dynamical mean-fieldtheory of strongly correlated fermion systems and the limit of infinite dimensions." \f{Rev. Mod. Phys} \bo{68}:13–125 (1996)}

\bibitem{fetter}
{A.L. Fetter and J.D. Walecka. "Quantum Theory of Many-Particle Systems." \f{Dover Publications, Inc} (2003)}

\bibitem{spektral}
{C. Titchmarsh. “The theory of functions.” (1939)}

\bibitem{sokhotsi}
{Peter Henrici. "Applied and Computational Complex Analysis, vol. 3." \f{Willey, John $\&$ Sons, Inc.} (1986)}

\bibitem{ladder}
{R. Mattuck. "A guide to feynman diagram in the many-body problems." (1992)}

\bibitem{hubbard}
{J. Hubbard. “Electron correlations in narrow energy bands.” \f{Proceedingsof the Royal Society of London. Series A. Mathematical and Physical Sciences}, \bo{vol. 276}, no. 1365, pp. 238–257, (1963)}

\bibitem{mott-transition0}
{H. Strand. “Critical properties of the mott-hubbard metal-insulatortransition.” (2011)}

\bibitem{magnetic-excitation}
{M. Karski and G. U. C. Raas. “Single-particle dynamics in the vicin-ity of the mott-hubbard metal-to-insulator transition,” \bo{vol. 77}, no.7,p. 075116, (2008)}

\bibitem{mott-transition1}
{ J. Joo and V. Oudovenko. "Quantum Monte Carlo calculation of the finite tem-perature Mott-Hubbard transition." \f{Phys. Rev. B},  \bo{64}:193102,(2001) }

\bibitem{mott-transition2}
{R. Bulla, T. A. Costi, and D. Vollhardt. "Finite-temperature numerical renormal-ization group study of the Mott transition." \f{Phys. Rev. B}, \bo{64}:045103, (2001)}

\bibitem{mott-transition3}
{M. Capone, L. de’ Medici, and A. Georges.  "Solving the dynamical mean-fieldtheory at very low temperatures using the Lanczos exact diagonalization." \f{Phys.Rev. B}, \bo{76}:245116, (2007)}

\bibitem{mott-transition4}
{H. Strand, A. Sabashvili, M. Granath, B. Hellsing, and S. Östlund.  "Dynamical mean field theory phase-space extension and critical properties of the finitetemperature Mott transition." \f{Phys. Rev. B}, \bo{83}:205136, (2011)}

\end{thebibliography}

