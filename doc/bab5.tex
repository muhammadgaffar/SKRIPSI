%-----------------------------------------------------------------------------%
\chapter{\babLima}
%-----------------------------------------------------------------------------%

%-----------------------------------------------------------------------------%
\section{Kesimpulan}
%-----------------------------------------------------------------------------%

Studi yang dilakukan dengan mengembangkan metode penyelesaian impuritas yang baru yakni pengembangan medan rata-rata yang melibatkan fluktuasi okupansi memberikan kesimpulan berikut:

\begin{itemize}
\item Pada kasus yang dipaksa paramagnetik, didapatkan bahwa fluktuasi okupansi tidak memberikan hasil fisis yang tepat. Hal ini dikarenakan dengan menekan atau menyamakan okupansi untuk kedua basis $\left\lbrace \uparrow, \downarrow\right\rbrace$ menjadikan fluktuasi okupansi sebagai kuantitas yang jika dirata-ratakan masih tetap bernilai disekitar nol, sehingga efek \textit{self-energy} tidak terlalu menampakkan dinamika fluktuasi yang sesuai, sebagaimana yang dicapai oleh iterasi pertubasi teori.

\item Pasa kasus keadaan yang memperbolehkan adanya keadaan antiferomagnetik, dapat dilihat bahwa fluktuasi okupansi memberikan hasil yang lebih baik dibanding medan rata-rata. Hal ini dilihat dari bagaimana perilaku DOS dan nilai \textit{self-energy}. Kontribusi \textit{self-energy} dapat dilihat berasal dari kontribusi fluktuasi okupansi. Namun jika pada $U$ rendah atau $T$ sangat tinggi, dimana hasil metode ini kembali menjadi paramagnetik, hasil yang diberikan kembali tidak terlalu sesuai dengan intepretasi fisis seharusnya. Sehingga fluktuasi okupansi tidak dapat digunakan secara umum untuk semua interval $U$ dan $T$.

\item Dengan menggunakan fluktuasi okupansi sebagai metode koreksi untuk IPT, didapatkan hasil yang menarik, dimana imajiner dari \textit{self-energy} diagram orde kedua IPT mengalami koeksistensi dengan kontribusi dari fluktuasi okupansi, dilihat terpisahnya dua \textit{self-energy} untuk basis yang berbeda. Sifat koeksistensi ini menarik dikarenakan pada IPT murni, terjadi kompetisi dalam pembentukan gap antara pengaruh perbedaan okupansi dan \textit{self-energy} orde kedua, sehingga terjadi transisi fasa magnetik secara mendadak. Namun, dengan memberikan Okupansi fluktuasi sebagai koreksi, transisi tidak terjadi secara mendadak, namun perlahan-lahan secara kontinu yang memperlihatkan adanya pengaruh dari fluktuasi okupansi itu sendiri.

\item Hasil utama fasa diagram magnetik 3D Kubik menunjukkan bahwa koreksi dari fluktuasi okupansi tidak menunjukkan adanya signifikansi yang berarti dalam mengoreksi IPT. Secara umum, informasi atau tebakan dari IPT masih sangat berpengaruh besar terhadap hasil akhir.

\item Secara numerik, penyelesaian menggunakan medan rata-rata memberikan \textit{CPU time} yang sangat cepat, lalu IPT, dan diakhir OF. Perbedaan ketiganya relatif tidak jauh, namun sumber daya komputasi OF sangat dipengaruhi oleh paramater jumlah titik fluktuasi okupansi yang dicacah, sehingga jika dibuat untuk akurasi yang tinggi, perhitungan komputasi menjadi tidak sangat efisien, berbeda dengan medan rata-rata dan IPT yang tidak memiliki pengaruh ini. Walaupun secara umum, OF masih cukup cepat dibandingkan metode penyelesaian impuritas yang lebih eksak seperti CTQMC.

\end{itemize}


%-----------------------------------------------------------------------------%
\section{Saran}
%-----------------------------------------------------------------------------%

Metode penyelesaian impuritas fluktuasi okupansi baik secara murni maupun koreksi IPT perlu untuk dievaluasi ulang, yakni dalam hal konsep. Hal ini dikarenakan penurunan metode ini tidak sepenuhnya berasal dari prinsip dasar. Terlebih, pengenalan fluktuasi kepada sistem dilakukan secara semiklasik, dimana fluktuasi diperkenalkan secara tiba-tiba dan fluktuasinya independen terhadap parameter fisis lainnya, yang dimana hanya dicacah dari nilai okupansi maksimum hingga minimum.

Secara numerik, juga kembali dipertimbangkan, dimana keberadaan kuantitas fluktuasi untuk setiap basis, mengakibatkan ledakan dimensional dalam menghitung probabilitas pada domain matsubara, terutama jika dilakukan pada sistem yang lebih real, dimana memiliki orbital yang sangat tinggi.