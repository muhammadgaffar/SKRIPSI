%-----------------------------------------------------------------------------%
\addChapter{Lampiran 1}
\chapter{Derivasi DOS dari kisi Bethe dan Kubik}

\section{Kisi Bethe}
Fungsi medan Weiss (kolam) $\mathcal{G}$ dapat dihubungkan dengan fungsi Green lokal\cite{DMFT}
\begin{align}
\mathcal{G}_0^{-1}(i\omega_n) = i\omega_n + \mu - \sum_{ij} t_{oi} t_{oj} G_{ij}^0
\end{align}
dengan hanya mempertimbangkan tetangga terdekat, dan energi kinetik \textit{hopping} terhadap skala dimensi adalah $t_{ij} = t / \sqrt{z}$, untuk jumlah konektivitas $z$ pada kisi bethe. Maka persamaan A.1 diatas menjadi (dengan di set $\zeta = i\omega_n + \mu$)
\begin{align}
G_{00}^{-1}(\zeta) = \zeta - \frac{t^2}{z}\sum_{i,o} G_{ii}^0 = \zeta - t^2G_{ii}^0
\end{align}
dengan menghilangkan kisi dengan indeks $0$, maka tetangga terdekatnya sekarang memiliki konektivitas sejumlah $z-1$. Maka fungsi Green untuk tetangga terdekat ini adalah
\begin{align}
[G^0_{ii}]^{-1} = \zeta - (z - 1)\frac{t^2}{z}G_{jj}^{o,i}
\end{align}
untuk kisi yang tak terbatas, maka terdapat translasi invarian dimana kisi $i$ sama dengan $j$, sehingga persamaan A.3 diatas menjadi
\begin{align}
\frac{z-1}{z}t^2[G^0_{ii}]^2 + \zeta G^0_{ii} + 1 = 0
\end{align}
untuk kondisi $\text{Im}(\zeta)$, maka $G^0_{ii}$ memiliki solusi
\begin{align}
G = \frac{(z-2)\zeta - z\sqrt{\zeta^2 - 4(z-1)t^2/z}}{2(z t^2 - \zeta^2}
\end{align}
dengan DOS $D(\omega) = -\text{Im} G(\omega + i0^+)$, didapat
\begin{align}
D(\omega) = \frac{\sqrt{\omega^2 - 4(z-1)t^2/z}}{2\pi (t^2 - \omega^2 /z)}
\end{align}
mengambil batas $z \rightarrow \infty$,
\begin{align}
D(\omega) = \frac{\sqrt{\omega^2 - 4t^2}}{2\pi t^2}
\end{align}

\section{Kisi Kubik}

Energi dispersi pada kubik diberikan dengan
\begin{align}
\epsilon_\bo{k} = -2t[\cos(k_x) + \cos(k_y) + \cos(k_z)]
\end{align}
maka DOS yang diberikan oleh energi dispersi ini adalah
\begin{align}
D(\omega	) &= \frac{1}{2\pi^3} \int_{-\pi}^\pi dk_x \int_{-\pi}^\pi dk_y \int_{-\pi}^\pi dk_z \delta(\omega - \epsilon_\bo{k})\\
&= \frac{1}{\pi^3 2t}\int_0^\pi dk_x \int_0^\pi dk_y \int_0^\pi dk_z  \delta(\tilde{\omega} + \cos(k_x) + \cos(k_y) + \cos(k_z))
\end{align}
dimana $\tilde{\omega} = \omega / 2t$. Dengan melakukan subtitusi variabel untuk setiap nilai $\cos(k_\alpha) = \alpha$ menjadi $\alpha = x,y,z$, maka didapat
\begin{align}
D(\omega) = \frac{1}{\pi^32t}\int_{-1}^{1} dx \int_{-1}^{1} dy \int_{-1}^{1} dz \frac{\delta(\tilde{\omega} + x + y + z)}{\sqrt{(1-x^2)(1-y^2)(1-z^2)}}
\end{align}
Integral fungsi Delta memilik properti
\begin{align}
\int_{-\infty}^\infty dx f(x) \delta(g(x)) = \sum_i \frac{f(x_i)}{\vert g'(x_i) \vert}; \quad \text{dengan} \; g(x_i) = 0
\end{align}
dengan mengintegrasi terhadap variabel $y$, sesuai dengan properti integral diatas, maka didapat
\begin{align}
D(\omega) = \frac{1}{\pi^3 2t}\int dx \int dz \frac{1}{\sqrt{(1-x^2)(1-y^2)[1-(\tilde{\omega} + x + z)^2]}}
\end{align}
dari relasi trigonometri
\begin{align}
\sin(\arccos(x)) = \cos(\arcsin(x)) = \sqrt{1 - x^2}
\end{align}
kita dapat menentukan batas integrasi pada persamaan A.13, dimana
\begin{align}
-1 \leq z &\leq 1, \; x_{min} \leq x \leq x_{max},\\
x_{min} &= \text{max}[-1,-1-(\tilde{\omega}+z)],\\
x_{max} &= \text{min}[1,1-(\tilde{\omega} + z)]
\end{align}
keterbergantungan terhadap $(\tilde{\omega} + z)$ membentuk daerah integrasi yang berbeda,
\begin{align}
I &: -1 < x < 1 - (\tilde{\omega} - z) | \; \text{untuk} \; (\tilde{\omega} + z) > 0 \\
II &: -1 - (\tilde{\omega} - z) < x < 1 | \; \text{untuk} \; (\tilde{\omega} + z) < 0
\end{align}
nilai akar $x$ pada polinomial persamaan A.13 adalah
\begin{align}
x_1 = 1, \; x_2 = -1, \; x_3 = -1 - \tilde{\omega} - z, \; x_4 = 1 - \tilde{\omega} - z 
\end{align}
maka persamaan A.13 dapat di simplikasi menjadi
\begin{align}
D(\omega) = \frac{1}{\pi^3 2t}\int dx \int dz \frac{1}{\sqrt{(1-z^2)\prod_{i=1}^4 ( x - x_i )}}
\end{align}
integrasi polinomial $x$,
\begin{align}
\int \frac{dx}{\sqrt{P(x)}}
\end{align}
dapat ditransformasi. Pertama-tama, kita lakukan transformasi variabel
\begin{align}
x(\phi) = \frac{\gamma(\beta - \delta) - \delta(\beta - \gamma)\sin^2(\phi)}{(\beta-\delta) - \delta(\beta - \gamma)\sin^2(\phi)}
\end{align}
dengan $\delta < \gamma < \beta < \alpha$ adalah akar-akar real dari persamaan polinomial $P(x)$. Ini dapat ditransformasi ke integral spesial yang disebut sebagai integral Elliptik jenis pertama,
\begin{align}
\int \frac{dx}{\sqrt{P(x)}} \rightarrow \frac{2}{\sqrt{(\alpha - \gamma)(\beta - \delta)}}\int \frac{d\phi}{\sqrt{1 - m\sin^2(\phi)}}
\end{align}
sehingga DOS kubik menjadi
\begin{align}
D(\omega) = \frac{1}{\pi^32t}\int \frac{dz}{\sqrt{1-z^2}}\frac{2}{\sqrt{(\alpha - \gamma)(\beta -\delta)}} \int \frac{d\phi}{\sqrt{1 - m\sin^2(\phi)}}
\end{align}
Sekarang kita lihat transformasi integral diatas untuk kedua daerah, untuk daerah pertama, urutan akar-akarnya adalah $x_1 < x_4 < x_2 < x_3$ sehingga $(\alpha - \gamma)(\beta -\delta) \rightarrow (x_3 - x_4)(x_2 - x_1)$, sedangkan untuk daerah kedua, $x_4 < x_1 < x_3 < x_2$ sehingga $(\alpha - \gamma)(\beta -\delta) \rightarrow (x_2 - x_1)(x_3 - x_4)$. Kedua daerah memberikan transformasi yang sama, maka didapat
\begin{align}
D(\omega) = \frac{1}{\pi^32t}\int \frac{dz}{\sqrt{1-z^2}} K(m), \quad m = 1 - \left( \frac{\tilde{\omega} + z}{2} \right)^2
\end{align}
dimana $K(m)$ adalah integral Eliptik jenis pertama, dimana
\begin{align}
K(m) = \int_0^{\pi/2} \frac{d\phi}{\sqrt{1 - m\sin^2(\phi)}}
\end{align}

%-----------------------------------------------------------------------------%

\chapter{\textit{Analytical Continuation aproksimasi IPT}}

Suku \textit{self-energy} pertubasi orde kedua ditulis
\begin{align}
\Sigma_{i\sigma}(i\omega_n) = U\left< n_{i\bar{\sigma}} \right> - \frac{U^2}{\beta^2} \sum_{m,p} \mathcal{G}_{i\sigma}(i\omega + i\nu_m)\mathcal{G}_{i\bar{\sigma}}(i\omega_p + i\nu_m)\mathcal{G}_{i\bar{\sigma}}(i\omega_p)
\end{align}
dari relasi dengan DOS
\begin{align}
G(i\omega_n) = \int d\omega \; \frac{D(\omega)}{i\omega_n - \omega}
\end{align}
dan mengkonstruski suku diagram \textit{bubble} sebagai
\begin{align}
\Pi(i\nu_m) &= \frac{1}{\beta} \sum_n \mathcal{G}(i\omega_n + i\nu_m)\mathcal{G}(i\omega_n)\notag\\
&= \int d\omega \int d\omega' D(\omega)D(\omega') \frac{1}{\beta} \sum_n \frac{1}{i\omega_n - \omega}\frac{1}{i\omega_n + i\nu_m - \omega'}\notag\\
&= \int d\omega \int d\omega' D(\omega)D(\omega')\frac{f(\omega) - f(\omega')}{i\omega_n + \omega - \omega'}
\end{align}
maka \textit{self-energy} orde kedua dapat ditulis
\begin{align}
\Sigma(i\omega_n) &= -\frac{U^2}{\beta^2}\sum_m \mathcal{G}(i\omega_n + i\nu_m)\Pi(i\nu_m)\notag\\
&= -U^2 \int d\omega \int d\omega' \int d\omega'' D(\omega)D(\omega')D(\omega'') \frac{f(-\omega)f(\omega')f(-\omega'') + f(\omega)f(-\omega')f(\omega'')}{i\omega_n - \omega + \omega' - \omega''}
\end{align}
dengan mengenal notasi
\begin{align}
D^+(\omega) &= f(\omega)D(\omega)\\
D^-(\omega) &= f(1 - \omega)D(\omega)
\end{align}
didapat
\begin{align}
\Sigma(\omega) = U\left< n_{i\bar{\sigma}} \right> - U^2 \int d\nu \int d\nu' \int d\nu'' \frac{[D^-(\nu)D^+(\nu')D^-(\nu'') + D^+(\nu)D^-(\nu')D^+(\nu'')]}{\omega + i0^+ - \nu + \nu' - \nu''}
\end{align}